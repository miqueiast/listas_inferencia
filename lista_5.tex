%%%%%%%%%%%%%%%%%%%%%%%%%%%%%%%%%%%%%%%%%%%%%%%%%%%%%%%%%%%%%%%%%%%%%%%%%%%%%%%%
% PREÂMBULO: Configurações do documento e pacotes
%%%%%%%%%%%%%%%%%%%%%%%%%%%%%%%%%%%%%%%%%%%%%%%%%%%%%%%%%%%%%%%%%%%%%%%%%%%%%%%%
\documentclass[12pt, a4paper]{article}

% Pacotes para acentuação e linguagem em português
\usepackage[utf8]{inputenc}
\usepackage[T1]{fontenc}
\usepackage[brazil]{babel}

% Pacotes para matemática
\usepackage{amsmath}
\usepackage{amssymb}

% Pacote para ajustar as margens da página
\usepackage[margin=2.5cm]{geometry}

% Pacote para incluir imagens (necessário para o logo)
\usepackage{graphicx}

% Melhora a aparência das fontes
\usepackage{lmodern}

% Pacote para tabelas com linhas horizontais bonitas
\usepackage{booktabs}

%%%%%%%%%%%%%%%%%%%%%%%%%%%%%%%%%%%%%%%%%%%%%%%%%%%%%%%%%%%%%%%%%%%%%%%%%%%%%%%%
% INÍCIO DO DOCUMENTO
%%%%%%%%%%%%%%%%%%%%%%%%%%%%%%%%%%%%%%%%%%%%%%%%%%%%%%%%%%%%%%%%%%%%%%%%%%%%%%%%
\begin{document}

%%%%%%%%%%%%%%%%%%%%%%%%%%%%%%%%%%%%%%%%%%%%%%%%%%%%%%%%%%%%%%%%%%%%%%%%%%%%%%%%
% PÁGINA DE ROSTO
%%%%%%%%%%%%%%%%%%%%%%%%%%%%%%%%%%%%%%%%%%%%%%%%%%%%%%%%%%%%%%%%%%%%%%%%%%%%%%%%
\begin{titlepage}
    \centering

    % Para o logo, salve a imagem como 'dest_logo.png' na mesma pasta do seu arquivo .tex
    % \includegraphics[width=0.4\textwidth]{dest_logo.png}\par\vspace{1cm}

    % Se não tiver a imagem, pode usar o texto:
    {\Huge \textbf{DEST}}\par

    \vspace{2cm}

    {\Large \textbf{Inferência Estatística}} \par
    {\Large Lista 4 } \par

    \vspace{2.5cm}

    {\Large \textbf{Lista de Exercícios 5 - Resolução}}

    \vfill % Empurra o texto seguinte para a parte de baixo da página

    \large
    \begin{flushleft}
    \textbf{Nome:} Miqueias T. \\
    \textbf{Data:} \today
    \end{flushleft}
\end{titlepage}



%%%%%%%%%%%%%%%%%%%%%%%%%%%%%%%%%%%%%%%%%%%%%%%%%%%%%%%%%%%%%%%%%%%%%%%%%%%%%%%%
% CORPO DO DOCUMENTO - QUESTÕES E RESOLUÇÕES
%%%%%%%%%%%%%%%%%%%%%%%%%%%%%%%%%%%%%%%%%%%%%%%%%%%%%%%%%%%%%%%%%%%%%%%%%%%%%%%%

\section*{Questão 1}
\textit{Um experimento genético envolve uma população de moscas de frutas que consiste em 1 macho (Mike) e 3 fêmeas, chamadas Ana, Bárbara e Cristina. Suponha que duas moscas de frutas sejam selecionadas aleatoriamente com reposição.}

\begin{itemize}
    \item[\textbf{a)}] \textit{Depois de listar as 16 diferentes amostras possíveis, ache a proporção de fêmeas em cada amostra e, então, use uma tabela para descrever a distribuição amostral da proporção de fêmeas.}
    \item[\textbf{b)}] \textit{Ache a média da distribuição amostral.}
    \item[\textbf{c)}] \textit{A média da distribuição amostral (item b) é igual à proporção populacional de fêmeas?}
\end{itemize}

\textbf{Solução:}
\begin{itemize}
    \item[\textbf{a)}] A população total é de 4 moscas (M, A, B, C). A amostragem é com reposição, então há $4 \times 4 = 16$ amostras possíveis. A tabela abaixo lista todas as amostras e a proporção de fêmeas ($\hat{p}$) em cada uma.

    \begin{center}
    \begin{tabular}{c|cccc}
        \toprule
        \textbf{1ª Mosca} & \multicolumn{4}{c}{\textbf{2ª Mosca}} \\
        \cmidrule{2-5}
        \textbf{(Amostra)} & \textbf{Macho (M)} & \textbf{Ana (A)} & \textbf{Bárbara (B)} & \textbf{Cristina (C)} \\
        \midrule
        \textbf{Macho (M)} & (M,M) $\to \hat{p}=0$ & (M,A) $\to \hat{p}=0.5$ & (M,B) $\to \hat{p}=0.5$ & (M,C) $\to \hat{p}=0.5$ \\
        \textbf{Ana (A)} & (A,M) $\to \hat{p}=0.5$ & (A,A) $\to \hat{p}=1$ & (A,B) $\to \hat{p}=1$ & (A,C) $\to \hat{p}=1$ \\
        \textbf{Bárbara (B)} & (B,M) $\to \hat{p}=0.5$ & (B,A) $\to \hat{p}=1$ & (B,B) $\to \hat{p}=1$ & (B,C) $\to \hat{p}=1$ \\
        \textbf{Cristina (C)} & (C,M) $\to \hat{p}=0.5$ & (C,A) $\to \hat{p}=1$ & (C,B) $\to \hat{p}=1$ & (C,C) $\to \hat{p}=1$ \\
        \bottomrule
    \end{tabular}
    \end{center}
    
    Contando as ocorrências de cada proporção:
    \begin{itemize}
        \item $\hat{p} = 0$: 1 vez
        \item $\hat{p} = 0.5$: 6 vezes
        \item $\hat{p} = 1$: 9 vezes
    \end{itemize}
    
    A distribuição amostral da proporção de fêmeas é:
    \begin{center}
    \begin{tabular}{cc}
        \toprule
        \textbf{Proporção Amostral ($\hat{p}$)} & \textbf{Probabilidade P($\hat{p}$)} \\
        \midrule
        0 & $1/16$ \\
        0.5 & $6/16 = 3/8$ \\
        1.0 & $9/16$ \\
        \bottomrule
    \end{tabular}
    \end{center}

    \item[\textbf{b)}] A média da distribuição amostral ($\mu_{\hat{p}}$) é o valor esperado de $\hat{p}$:
    $$ \mu_{\hat{p}} = E[\hat{p}] = \sum \hat{p} \cdot P(\hat{p}) $$
    $$ \mu_{\hat{p}} = \left(0 \cdot \frac{1}{16}\right) + \left(0.5 \cdot \frac{6}{16}\right) + \left(1.0 \cdot \frac{9}{16}\right) $$
    $$ \mu_{\hat{p}} = 0 + \frac{3}{16} + \frac{9}{16} = \frac{12}{16} = \frac{3}{4} = 0.75 $$

    \item[\textbf{c)}] A proporção populacional de fêmeas ($p$) é de 3 fêmeas para 4 moscas no total, logo $p = 3/4 = 0.75$.
    
    \textbf{Sim}, a média da distribuição amostral ($\mu_{\hat{p}} = 0.75$) é exatamente igual à proporção populacional ($p = 0.75$). Isso ocorre porque a proporção amostral $\hat{p}$ é um estimador não viciado da proporção populacional $p$.
\end{itemize}

\newpage

\section*{Questão 2}
\textit{As idades (anos) dos quatro presidentes dos Estados Unidos quando foram assassinados no exercício do cargo são 56 (Lincoln), 49 (Garfield), 58 (McKinley) e 46 (Kennedy).}
\begin{itemize}
    \item[\textbf{a)}] \textit{Supondo que duas das idades sejam selecionadas com reposição, liste as 16 diferentes amostras possíveis.}
    \item[\textbf{b)}] \textit{Ache a média de cada uma das 16 amostras e, então, resuma a distribuição amostral das médias no formato de uma tabela que represente uma distribuição de probabilidade.}
\end{itemize}

\textbf{Solução:}
\begin{itemize}
    \item[\textbf{a)}] As 16 amostras possíveis de tamanho 2, com reposição, são:
    \begin{center}
    \begin{tabular}{cccc}
        \toprule
        (56, 56) & (56, 49) & (56, 58) & (56, 46) \\
        (49, 56) & (49, 49) & (49, 58) & (49, 46) \\
        (58, 56) & (58, 49) & (58, 58) & (58, 46) \\
        (46, 56) & (46, 49) & (46, 58) & (46, 46) \\
        \bottomrule
    \end{tabular}
    \end{center}
    
    \item[\textbf{b)}] Primeiro, calculamos a média ($\bar{x}$) para cada uma das 16 amostras:
    \begin{center}
    \begin{tabular}{cccc}
        \toprule
        56.0 & 52.5 & 57.0 & 51.0 \\
        52.5 & 49.0 & 53.5 & 47.5 \\
        57.0 & 53.5 & 58.0 & 52.0 \\
        51.0 & 47.5 & 52.0 & 46.0 \\
        \bottomrule
    \end{tabular}
    \end{center}
    
    Agora, resumimos a distribuição amostral das médias, agrupando os valores idênticos e calculando suas probabilidades (frequência/16).
    \begin{center}
    \begin{tabular}{ccc}
        \toprule
        \textbf{Média Amostral ($\bar{x}$)} & \textbf{Frequência} & \textbf{Probabilidade P($\bar{x}$)} \\
        \midrule
        46.0 & 1 & $1/16$ \\
        47.5 & 2 & $2/16 = 1/8$ \\
        49.0 & 1 & $1/16$ \\
        51.0 & 2 & $2/16 = 1/8$ \\
        52.0 & 2 & $2/16 = 1/8$ \\
        52.5 & 2 & $2/16 = 1/8$ \\
        53.5 & 2 & $2/16 = 1/8$ \\
        56.0 & 1 & $1/16$ \\
        57.0 & 2 & $2/16 = 1/8$ \\
        58.0 & 1 & $1/16$ \\
        \midrule
        \textbf{Total} & \textbf{16} & \textbf{1.0} \\
        \bottomrule
    \end{tabular}
    \end{center}
\end{itemize}

\section*{Questão 3}
\textit{Repita o Exe 2 usando a mediana em lugar de médias.}

\textbf{Solução:}
O procedimento é análogo ao do Exercício 2, mas em vez da média, calculamos a mediana para cada uma das 16 amostras. As amostras possíveis são as mesmas.

É importante notar que para uma amostra de tamanho n=2, com valores $\{x_1, x_2\}$, a mediana é definida como a média dos dois valores centrais, ou seja, $(x_1 + x_2) / 2$. Este cálculo é \textbf{idêntico ao da média} para uma amostra de tamanho 2. Portanto, os valores numéricos das medianas amostrais serão os mesmos das médias amostrais do exercício anterior.

Primeiro, calculamos a mediana para cada uma das 16 amostras:
\begin{center}
\begin{tabular}{cccc}
    \toprule
    56.0 & 52.5 & 57.0 & 51.0 \\
    52.5 & 49.0 & 53.5 & 47.5 \\
    57.0 & 53.5 & 58.0 & 52.0 \\
    51.0 & 47.5 & 52.0 & 46.0 \\
    \bottomrule
\end{tabular}
\end{center}

Em seguida, resumimos a distribuição amostral das medianas, agrupando os valores idênticos e calculando suas probabilidades. Como os valores são os mesmos do exercício anterior, a distribuição de probabilidade também será idêntica.

\begin{center}
\begin{tabular}{ccc}
    \toprule
    \textbf{Mediana Amostral} & \textbf{Frequência} & \textbf{Probabilidade} \\
    \midrule
    46.0 & 1 & $1/16$ \\
    47.5 & 2 & $2/16 = 1/8$ \\
    49.0 & 1 & $1/16$ \\
    51.0 & 2 & $2/16 = 1/8$ \\
    52.0 & 2 & $2/16 = 1/8$ \\
    52.5 & 2 & $2/16 = 1/8$ \\
    53.5 & 2 & $2/16 = 1/8$ \\
    56.0 & 1 & $1/16$ \\
    57.0 & 2 & $2/16 = 1/8$ \\
    58.0 & 1 & $1/16$ \\
    \midrule
    \textbf{Total} & \textbf{16} & \textbf{1.0} \\
    \bottomrule
\end{tabular}
\end{center}

\section*{Questão 4}
\textit{Considere o seguinte problema (adaptado de Magalhães \& Lima, 2006): Um fabricante afirma que sua vacina contra gripe imuniza em 80\% dos casos. Uma amostra de 25 indivíduos entre os que tomaram a vacina foi sorteada e testes foram feitos para verificar a imunização ou não desses indivíduos.}
\begin{itemize}
    \item[\textbf{a)}] \textit{No contexto do problema identifique: a população, o parâmetro de interesse, o estimador, a estimativa, a distribuição amostral.}
    \item[\textbf{b)}] \textit{Se o fabricante estiver correto, qual é a probabilidade da proporção de imunizados na amostra ser inferior a 0,75? E superior a 0,85?}
\end{itemize}

\textbf{Solução:}
\begin{itemize}
    \item[\textbf{a)}] Identificação dos elementos estatísticos:
    \begin{itemize}
        \item \textbf{População:} O conjunto de todos os indivíduos que tomaram ou poderiam tomar a vacina.
        
        \item \textbf{Parâmetro de interesse:} A verdadeira proporção populacional ($p$) de indivíduos que são imunizados pela vacina. O valor afirmado pelo fabricante é $p = 0.80$.
        
        \item \textbf{Estimador:} É a estatística amostral usada para estimar o parâmetro $p$. Neste caso, é a proporção amostral, representada por $\hat{p} = \frac{X}{n}$, onde $X$ é o número de indivíduos imunizados na amostra e $n$ é o tamanho da amostra.
        
        \item \textbf{Estimativa:} É o valor numérico do estimador obtido a partir de uma amostra específica. O problema não fornece o resultado da amostra, mas se, por exemplo, 21 dos 25 indivíduos fossem imunizados, a estimativa seria $\hat{p} = \frac{21}{25} = 0.84$.
        
        \item \textbf{Distribuição Amostral:} É a distribuição de probabilidade de todas as possíveis proporções amostrais ($\hat{p}$) que poderiam ser obtidas de amostras de tamanho $n=25$. Como a variável de interesse (ser ou não imunizado) é binária, o número de sucessos $X$ em $n$ ensaios segue uma distribuição Binomial, $X \sim B(n, p)$. A distribuição de $\hat{p}$ é, portanto, diretamente derivada da Binomial.
    \end{itemize}

    \item[\textbf{b)}] Para resolver este item, assumimos que a afirmação do fabricante é verdadeira, ou seja, $p=0.80$ e $n=25$. A variável $X$ (número de imunizados na amostra) segue $X \sim B(25, 0.80)$.
    
    Como o cálculo exato com a Binomial pode ser trabalhoso, podemos usar a \textbf{aproximação da Binomial pela Normal}, pois as condições são satisfeitas:
    $$ np = 25 \cdot 0.80 = 20 \ge 5 $$
    $$ n(1-p) = 25 \cdot 0.20 = 5 \ge 5 $$
    
    Calculamos a média ($\mu$) e o desvio padrão ($\sigma$) da distribuição de $X$:
    $$ \mu = np = 20 $$
    $$ \sigma = \sqrt{np(1-p)} = \sqrt{25 \cdot 0.80 \cdot 0.20} = \sqrt{4} = 2 $$
    
    Agora, convertemos as proporções em número de indivíduos ($X = n \cdot \hat{p}$):
    \begin{itemize}
        \item \textbf{Probabilidade de $\hat{p} < 0.75$:}
        Corresponde a $X < 25 \cdot 0.75 = 18.75$. Como $X$ é um número inteiro, isso é $P(X \le 18)$.
        Usando a \textbf{correção de continuidade}, aproximamos por $P(X < 18.5)$.
        Padronizando (cálculo do Z-score):
        $$ Z = \frac{18.5 - \mu}{\sigma} = \frac{18.5 - 20}{2} = \frac{-1.5}{2} = -0.75 $$
        A probabilidade é $P(Z < -0.75)$. Pela tabela da Normal Padrão, isso é aproximadamente \textbf{0.2266}.
        
        \item \textbf{Probabilidade de $\hat{p} > 0.85$:}
        Corresponde a $X > 25 \cdot 0.85 = 21.25$. Como $X$ é um número inteiro, isso é $P(X \ge 22)$.
        Usando a \textbf{correção de continuidade}, aproximamos por $P(X > 21.5)$.
        Padronizando:
        $$ Z = \frac{21.5 - \mu}{\sigma} = \frac{21.5 - 20}{2} = \frac{1.5}{2} = 0.75 $$
        A probabilidade é $P(Z > 0.75)$, que é $1 - P(Z < 0.75)$. Pela tabela da Normal Padrão, isso é $1 - 0.7734 = \textbf{0.2266}$.
    \end{itemize}
    Portanto, a probabilidade da proporção amostral ser inferior a 0,75 é de aproximadamente 22,66\%, e a probabilidade de ser superior a 0,85 também é de aproximadamente 22,66\%.
    
\end{itemize}

\section*{Questão 5}
\textit{Uma variável aleatória Y tem distribuição normal, com média 100 e desvio padrão 10.}
\begin{itemize}
    \item[\textbf{a)}] \textit{Qual a P(90 < Y < 110)?}
    \item[\textbf{b)}] \textit{Se $\bar{Y}$ for a média de uma amostra de 16 elementos retirados dessa população, calcule P(90 < $\bar{Y}$ < 110).}
\end{itemize}

\textbf{Solução:}
A variável aleatória da população, $Y$, segue uma distribuição Normal com $\mu=100$ e $\sigma=10$. Escrevemos: $Y \sim N(100, 10^2)$.

\begin{itemize}
    \item[\textbf{a)}] Para encontrar a probabilidade de uma única observação $Y$, primeiro padronizamos os valores 90 e 110 para a escala Z (Normal Padrão), usando a fórmula $Z = (Y - \mu) / \sigma$.
    
    Para $Y = 90$:
    $$ Z_1 = \frac{90 - 100}{10} = \frac{-10}{10} = -1 $$
    
    Para $Y = 110$:
    $$ Z_2 = \frac{110 - 100}{10} = \frac{10}{10} = 1 $$
    
    A probabilidade desejada é $P(-1 < Z < 1)$. Utilizando a tabela da distribuição Normal Padrão:
    $$ P(-1 < Z < 1) = P(Z < 1) - P(Z < -1) $$
    $$ P(-1 < Z < 1) = 0.8413 - 0.1587 = 0.6826 $$
    
    Portanto, $P(90 < Y < 110) = 0.6826$, ou 68,26\%. Este é o resultado esperado pela Regra Empírica para um desvio padrão em torno da média.
    
    \item[\textbf{b)}] Agora, estamos interessados na distribuição da \textbf{média amostral}, $\bar{Y}$, para amostras de tamanho $n=16$.
    
    Pelo Teorema Central do Limite (e como a população já é Normal), a distribuição de $\bar{Y}$ também é Normal. A média da distribuição amostral é a mesma da população ($\mu_{\bar{Y}} = \mu = 100$), mas o desvio padrão, chamado de \textbf{erro padrão}, é menor.
    
    O erro padrão é calculado como:
    $$ \sigma_{\bar{Y}} = \frac{\sigma}{\sqrt{n}} = \frac{10}{\sqrt{16}} = \frac{10}{4} = 2.5 $$
    
    Então, a distribuição da média amostral é $\bar{Y} \sim N(100, 2.5^2)$.
    
    Agora, padronizamos os valores 90 e 110 usando os parâmetros desta nova distribuição:
    
    Para $\bar{Y} = 90$:
    $$ Z_1 = \frac{90 - 100}{2.5} = \frac{-10}{2.5} = -4 $$
    
    Para $\bar{Y} = 110$:
    $$ Z_2 = \frac{110 - 100}{2.5} = \frac{10}{2.5} = 4 $$
    
    A probabilidade desejada é $P(-4 < Z < 4)$. Este intervalo contém quase toda a área da curva Normal.
    $$ P(-4 < Z < 4) = P(Z < 4) - P(Z < -4) $$
    $$ P(-4 < Z < 4) \approx 0.999968 - 0.000032 = 0.999936 $$
    
    Portanto, $P(90 < \bar{Y} < 110) \approx 0.9999$, ou 99,99\%.
\end{itemize}

\textbf{Comparação:} Note como a probabilidade aumentou drasticamente da parte (a) para a (b). Isso ocorre porque a distribuição das médias amostrais é muito mais "estreita" (tem menor variabilidade) do que a distribuição dos dados individuais. É muito mais provável que a média de 16 observações esteja perto da média populacional do que uma única observação aleatória.

\section*{Questão 6}
\textit{Utilizando algum recurso computacional ou tabela, calcule as probabilidades a seguir, conforme a distribuição da v.a. Y:}

\textbf{Solução:}
\vspace{0.5cm}

%---------------------------------------------------
% Distribuição t de Student
%---------------------------------------------------
\subsection*{\underline{Distribuição t de Student ($Y \sim t_{20}$)}}
\begin{itemize}
    \item $P(-2.85 \le Y \le 2.85) = P(Y \le 2.85) - P(Y \le -2.85) = 0.9950 - 0.0050 = \mathbf{0.990}$
    \item $P(Y < -2.85) = \mathbf{0.0050}$
    \item $P(Y > 2.85) = 1 - P(Y \le 2.85) = 1 - 0.9950 = \mathbf{0.0050}$
    \item $P(Y > 2.12) = 1 - P(Y \le 2.12) = 1 - 0.9774 = \mathbf{0.0226}$
    \item $P(Y < -3.01) = \mathbf{0.0034}$
\end{itemize}
\vspace{0.5cm}

%---------------------------------------------------
% Distribuição Qui-Quadrado
%---------------------------------------------------
\subsection*{\underline{Distribuição Qui-Quadrado ($Y \sim \chi^2_{16}$)}}
\begin{itemize}
    \item $P(8.91 < Y < 32.85) = P(Y < 32.85) - P(Y < 8.91) = 0.9925 - 0.0916 = \mathbf{0.9009}$
    \item $P(Y > 8.91) = 1 - P(Y \le 8.91) = 1 - 0.0916 = \mathbf{0.9084}$
    \item $P(Y > 32.85) = 1 - P(Y \le 32.85) = 1 - 0.9925 = \mathbf{0.0075}$
    \item $P(Y > 22.80) = 1 - P(Y \le 22.80) = 1 - 0.8756 = \mathbf{0.1244}$
    \item $P(Y < 10.12) = \mathbf{0.1558}$
\end{itemize}
\vspace{0.5cm}

%---------------------------------------------------
% Distribuição F de Snedecor
%---------------------------------------------------
\subsection*{\underline{Distribuição F de Snedecor ($Y \sim F_{(10,7)}$)}}
\begin{itemize}
    \item $P(Y > 3.18) = 1 - P(Y \le 3.18) = 1 - 0.9419 = \mathbf{0.0581}$
    \item $P(Y > 0.15) = 1 - P(Y \le 0.15) = 1 - 0.0016 = \mathbf{0.9984}$
    \item $P(Y > 5.35) = 1 - P(Y \le 5.35) = 1 - 0.9928 = \mathbf{0.0072}$
    \item $P(Y < 7.41) = \mathbf{0.9986}$
    \item $P(Y < 1) = \mathbf{0.4815}$
\end{itemize}
\vspace{1cm}


\section*{Questão 7}
\textit{Para cada uma das 3 distribuições propostas no exercício 6, encontre o valor de y tal que:}
\begin{itemize}
    \item[\textit{a)}] $P(Y < y) = 0.90$
    \item[\textit{b)}] $P(Y < y) = 0.025$
    \item[\textit{c)}] $P(Y < y) = 0.01$
    \item[\textit{d)}] $P(Y > y) = 0.975$
\end{itemize}

\textbf{Solução:}
Neste exercício, faremos o processo inverso (cálculo de quantis). Para o item (d), note que $P(Y > y) = 0.975$ é o mesmo que $P(Y < y) = 1 - 0.975 = 0.025$. Portanto, a resposta do item (d) é a mesma do item (b) para todas as distribuições.
\vspace{0.5cm}

%---------------------------------------------------
% Distribuição t de Student
%---------------------------------------------------
\subsection*{\underline{Distribuição t de Student ($Y \sim t_{20}$)}}
\begin{itemize}
    \item[a)] $P(Y < y) = 0.90 \implies y = \mathbf{1.325}$
    \item[b)] $P(Y < y) = 0.025 \implies y = \mathbf{-2.086}$
    \item[c)] $P(Y < y) = 0.01 \implies y = \mathbf{-2.528}$
    \item[d)] $P(Y > y) = 0.975 \implies P(Y < y) = 0.025 \implies y = \mathbf{-2.086}$
\end{itemize}
\vspace{0.5cm}

%---------------------------------------------------
% Distribuição Qui-Quadrado
%---------------------------------------------------
\subsection*{\underline{Distribuição Qui-Quadrado ($Y \sim \chi^2_{16}$)}}
\begin{itemize}
    \item[a)] $P(Y < y) = 0.90 \implies y = \mathbf{23.542}$
    \item[b)] $P(Y < y) = 0.025 \implies y = \mathbf{6.908}$
    \item[c)] $P(Y < y) = 0.01 \implies y = \mathbf{5.812}$
    \item[d)] $P(Y > y) = 0.975 \implies P(Y < y) = 0.025 \implies y = \mathbf{6.908}$
\end{itemize}
\vspace{0.5cm}

%---------------------------------------------------
% Distribuição F de Snedecor
%---------------------------------------------------
\subsection*{\underline{Distribuição F de Snedecor ($Y \sim F_{(10,7)}$)}}
\begin{itemize}
    \item[a)] $P(Y < y) = 0.90 \implies y = \mathbf{2.624}$
    \item[b)] $P(Y < y) = 0.025 \implies y = \mathbf{0.306}$
    \item[c)] $P(Y < y) = 0.01 \implies y = \mathbf{0.244}$
    \item[d)] $P(Y > y) = 0.975 \implies P(Y < y) = 0.025 \implies y = \mathbf{0.306}$
\end{itemize}

%%%%%%%%%%%%%%%%%%%%%%%%%%%%%%%%%%%%%%%%%%%%%%%%%%%%%%%%%%%%%%%%%%%%%%%%%%%%%%%%

\section*{Questão 8}
\textit{A máquina de empacotar um determinado produto o faz segundo uma distribuição normal, com média µ e desvio padrão 10 g.}
\begin{itemize}
    \item[\textbf{a)}] \textit{Em quanto deve ser regulado o peso médio µ para que apenas 10\% dos pacotes tenham menos do que 500 g?}
    \item[\textbf{b)}] \textit{Com a máquina assim regulada, qual a probabilidade de que o peso total de 4 pacotes escolhidos ao acaso seja inferior a 2 kg?}
\end{itemize}

\textbf{Solução:}
Seja $Y$ o peso de um pacote. A distribuição é Normal, com desvio padrão conhecido $\sigma = 10$ g. Escrevemos $Y \sim N(\mu, 10^2)$.

\begin{itemize}
    \item[\textbf{a)}] Queremos encontrar o valor da média $\mu$ tal que a probabilidade de um pacote ter menos de 500 g seja de 10\%. Matematicamente:
    $$ P(Y < 500) = 0.10 $$
    
    Para resolver, primeiro padronizamos a variável, transformando-a em uma Normal Padrão ($Z$):
    $$ P\left( \frac{Y - \mu}{\sigma} < \frac{500 - \mu}{10} \right) = 0.10 $$
    $$ P\left( Z < \frac{500 - \mu}{10} \right) = 0.10 $$
    
    Agora, precisamos encontrar o valor de $Z$ na tabela da Normal Padrão que acumula 10\% (ou 0.10) de probabilidade à sua esquerda. Este valor é o quantil $z_{0.10}$.
    Consultando a tabela (ou usando um recurso computacional), encontramos que $z_{0.10} \approx -1.2816$.
    
    Com isso, podemos igualar as expressões e resolver para $\mu$:
    $$ \frac{500 - \mu}{10} = -1.2816 $$
    $$ 500 - \mu = 10 \times (-1.2816) $$
    $$ 500 - \mu = -12.816 $$
    $$ \mu = 500 + 12.816 = 512.816 $$
    
    Portanto, o peso médio $\mu$ deve ser regulado em aproximadamente \textbf{512,82 g}.
    
    \item[\textbf{b)}] Usando a média regulada em $\mu = 512.82$ g, queremos a probabilidade de que o \textbf{peso total} de 4 pacotes ($n=4$) seja inferior a 2 kg.
    
    Primeiro, garantimos a consistência das unidades: 2 kg = 2000 g.
    
    A pergunta sobre o peso total ($S_4$) pode ser convertida em uma pergunta sobre a média amostral ($\bar{Y}$):
    $$ P(S_4 < 2000) \implies P(4 \cdot \bar{Y} < 2000) \implies P(\bar{Y} < \frac{2000}{4}) \implies P(\bar{Y} < 500) $$
    
    Precisamos da distribuição da média amostral $\bar{Y}$. Como a população é Normal, $\bar{Y}$ também segue uma distribuição Normal, com os seguintes parâmetros:
    \begin{itemize}
        \item Média: $\mu_{\bar{Y}} = \mu = 512.82$ g
        \item Erro padrão: $\sigma_{\bar{Y}} = \frac{\sigma}{\sqrt{n}} = \frac{10}{\sqrt{4}} = \frac{10}{2} = 5$ g
    \end{itemize}
    
    Assim, $\bar{Y} \sim N(512.82, 5^2)$. Agora, padronizamos o valor 500 para encontrar a probabilidade:
    $$ Z = \frac{500 - \mu_{\bar{Y}}}{\sigma_{\bar{Y}}} = \frac{500 - 512.82}{5} = \frac{-12.82}{5} = -2.564 $$
    
    A probabilidade desejada é $P(Z < -2.564)$. Consultando a tabela da Normal Padrão:
    $$ P(Z < -2.564) \approx 0.0052 $$
    
    A probabilidade de que o peso total de 4 pacotes seja inferior a 2 kg é de \textbf{0,52\%}.
    
\end{itemize}

\section*{Questão 9}
\textit{Um estudo que investiga a relação entre idade e despesas médicas anuais amostra aleatoriamente 100 indivíduos em uma cidade da Califórnia. Espera-se que a amostra tenha uma média de idade semelhante à de toda a população.}
\begin{itemize}
    \item[\textbf{a)}] \textit{Se o desvio padrão das idades de todos os indivíduos em Davis for $\sigma = 15$, encontre a probabilidade de que a idade média dos indivíduos da amostra esteja dentro de dois anos da idade média de todos os indivíduos na cidade. (Dica: encontre a distribuição amostral da idade média da amostra e use o teorema do limite central. Você não precisa saber a média da população para responder, mas se isso facilitar, use um valor como $\mu = 30$.)}
    \item[\textbf{b)}] \textit{A probabilidade seria maior ou menor se $\sigma = 10$? Por quê?}
\end{itemize}

\textbf{Solução:}
Seja $\bar{Y}$ a idade média da amostra e $\mu$ a idade média da população. Temos uma amostra de tamanho $n=100$.

\begin{itemize}
    \item[\textbf{a)}] Queremos encontrar a probabilidade de que a idade média da amostra ($\bar{Y}$) esteja "dentro de dois anos" da média da população ($\mu$). Matematicamente, isso significa que a diferença absoluta entre elas é menor ou igual a 2:
    $$ P(|\bar{Y} - \mu| \le 2) $$
    Isso é equivalente a $P(-2 \le \bar{Y} - \mu \le 2)$.

    Como o tamanho da amostra $n=100$ é grande, pelo \textbf{Teorema Central do Limite (TCL)}, a distribuição da média amostral $\bar{Y}$ é aproximadamente Normal. Os parâmetros dessa distribuição são:
    \begin{itemize}
        \item Média: $\mu_{\bar{Y}} = \mu$ (a própria média da população)
        \item Erro padrão: $\sigma_{\bar{Y}} = \frac{\sigma}{\sqrt{n}}$
    \end{itemize}
    
    Com $\sigma=15$, calculamos o erro padrão:
    $$ \sigma_{\bar{Y}} = \frac{15}{\sqrt{100}} = \frac{15}{10} = 1.5 \text{ anos} $$
    
    Para encontrar a probabilidade, padronizamos a inequação dividindo pelo erro padrão. Note que não precisamos saber o valor de $\mu$:
    $$ P\left( \frac{-2}{1.5} \le \frac{\bar{Y} - \mu}{1.5} \le \frac{2}{1.5} \right) $$
    O termo central, $\frac{\bar{Y} - \mu}{\sigma_{\bar{Y}}}$, é a nossa variável Z padronizada.
    $$ P(-1.33 \le Z \le 1.33) $$
    
    Usando a tabela da Normal Padrão:
    $$ P(Z \le 1.33) - P(Z \le -1.33) = 0.9082 - 0.0918 = 0.8164 $$
    
    A probabilidade de que a idade média da amostra esteja dentro de 2 anos da média da população é de \textbf{81,64\%}.

    \item[\textbf{b)}] A probabilidade seria \textbf{maior}.
    
    \textbf{Cálculo:} Se o desvio padrão da população fosse menor, $\sigma=10$, o erro padrão também seria menor:
    $$ \sigma_{\bar{Y}} = \frac{10}{\sqrt{100}} = \frac{10}{10} = 1.0 \text{ ano} $$
    
    A nova probabilidade seria:
    $$ P\left( \frac{-2}{1.0} \le Z \le \frac{2}{1.0} \right) = P(-2 \le Z \le 2) $$
    $$ P(Z \le 2) - P(Z \le -2) = 0.9772 - 0.0228 = 0.9544 $$
    A nova probabilidade seria de \textbf{95,44\%}, que é maior que a anterior.

    \textbf{Por quê?} Um desvio padrão populacional ($\sigma$) menor significa que os dados da população são menos dispersos e estão, em geral, mais próximos da média $\mu$. Consequentemente, o erro padrão ($\sigma_{\bar{Y}}$) também diminui, o que torna a distribuição das médias amostrais mais "estreita" e concentrada em torno de $\mu$. Com uma distribuição mais concentrada, a probabilidade de uma média amostral cair em um intervalo fixo (como $\pm 2$ anos) ao redor do centro aumenta.
    
\end{itemize}

\section*{Questão 10}
\textit{O teste de conhecimentos gerais chamado Graduate Record Examination (GRE) tem componentes que medem o raciocínio verbal e o raciocínio quantitativo. O exame verbal e o exame quantitativo têm cada um uma pontuação mínima de 200 e máxima de 800. Nos últimos anos, a pontuação total nos dois exames teve aproximadamente uma distribuição normal com uma média de cerca de 1050 e desvio padrão de cerca de 200.}
\begin{itemize}
    \item[\textbf{a)}] \textit{Qual a probabilidade de obter pontuação total (i) abaixo de 1200 e (ii) acima de 1200?}
    \item[\textbf{b)}] \textit{Dos participantes do teste GRE que pontuaram acima de 1.200, qual proporção deles teve pontuação acima de 1.400?}
    \item[\textbf{c)}] \textit{Um grupo de 25 alunos formou um grupo de estudos para se preparar para o GRE. Para eles, a média de suas 25 pontuações totais é 1200. Se eles fossem uma amostra aleatória dos alunos que estão fazendo o exame, explique por que isso teria sido um resultado muito incomum.}
\end{itemize}

\textbf{Solução:}
Seja $Y$ a pontuação total no GRE. A distribuição é Normal, com média $\mu = 1050$ e desvio padrão $\sigma = 200$. Logo, $Y \sim N(1050, 200^2)$.

\begin{itemize}
    \item[\textbf{a)}] Para encontrar as probabilidades, primeiro padronizamos o valor de 1200:
    $$ Z = \frac{Y - \mu}{\sigma} = \frac{1200 - 1050}{200} = \frac{150}{200} = 0.75 $$
    \begin{itemize}
        \item[\textbf{(i)}] A probabilidade de obter pontuação abaixo de 1200 é $P(Y < 1200) = P(Z < 0.75)$.
        Consultando a tabela da Normal Padrão, encontramos $\mathbf{0.7734}$.
        
        \item[\textbf{(ii)}] A probabilidade de obter pontuação acima de 1200 é $P(Y > 1200) = P(Z > 0.75)$.
        Isso é igual a $1 - P(Z \le 0.75) = 1 - 0.7734 = \mathbf{0.2266}$.
    \end{itemize}

    \item[\textbf{b)}] Esta é uma questão de probabilidade condicional. Queremos encontrar $P(Y > 1400 \mid Y > 1200)$.
    A fórmula é:
    $$ P(Y > 1400 \mid Y > 1200) = \frac{P(Y > 1400 \text{ e } Y > 1200)}{P(Y > 1200)} = \frac{P(Y > 1400)}{P(Y > 1200)} $$
    
    Já temos o denominador do item (a): $P(Y > 1200) \approx 0.2266$.
    
    Agora, calculamos o numerador, $P(Y > 1400)$. Primeiro, padronizamos 1400:
    $$ Z = \frac{1400 - 1050}{200} = \frac{350}{200} = 1.75 $$
    Então, $P(Y > 1400) = P(Z > 1.75) = 1 - P(Z \le 1.75) = 1 - 0.9599 = 0.0401$.
    
    Finalmente, calculamos a proporção:
    $$ \frac{0.0401}{0.2266} \approx 0.1769 $$
    
    Portanto, aproximadamente \textbf{17,69\%} dos participantes que pontuaram acima de 1200 também tiveram pontuação acima de 1400.

    \item[\textbf{c)}] Para avaliar se uma média amostral de $\bar{Y}=1200$ para $n=25$ alunos é "incomum", calculamos a probabilidade de obter uma média amostral \textbf{tão alta ou maior} por puro acaso.
    
    Primeiro, definimos a distribuição da média amostral $\bar{Y}$. Como a população é Normal, $\bar{Y}$ também é Normal.
    \begin{itemize}
        \item Média: $\mu_{\bar{Y}} = \mu = 1050$
        \item Erro padrão: $\sigma_{\bar{Y}} = \frac{\sigma}{\sqrt{n}} = \frac{200}{\sqrt{25}} = \frac{200}{5} = 40$
    \end{itemize}
    
    A distribuição é $\bar{Y} \sim N(1050, 40^2)$. Agora, calculamos $P(\bar{Y} \ge 1200)$. Padronizamos o valor 1200:
    $$ Z = \frac{\bar{Y} - \mu_{\bar{Y}}}{\sigma_{\bar{Y}}} = \frac{1200 - 1050}{40} = \frac{150}{40} = 3.75 $$
    
    A probabilidade é $P(Z \ge 3.75)$, que é $1 - P(Z < 3.75)$.
    
    Um valor Z de 3.75 é extremamente alto. A probabilidade associada a ele é minúscula:
    $$ P(Z \ge 3.75) \approx 1 - 0.999912 = 0.000088 $$
    
    \textbf{Explicação:} A probabilidade de uma amostra aleatória de 25 alunos ter uma nota média de 1200 ou mais é de apenas 0,0088\%. Como essa probabilidade é extremamente baixa, o resultado é considerado muito incomum. Isso sugere fortemente que o grupo de estudos não se comporta como uma amostra aleatória da população geral; o desempenho deles é significativamente superior, seja por serem alunos naturalmente mais aptos ou pelo efeito positivo do grupo de estudos.
    
\end{itemize}

\end{document}